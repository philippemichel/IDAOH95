%
%  Abbréviations Statistiques
%
\newabbreviation{anova}{anova}{analysis of variance}
\newabbreviation{aic}{aic}{Akaike information criterion}
%
% Abbréviations médicales
%

\newabbreviation{ph}{ph}{praticien hospitalier}
\newabbreviation{ch}{ch}{centre hospitalier}
\newabbreviation{naco}{naco}{Nouveaux anticoagulants oraux}
\newabbreviation{idec}{idec}{infirmier(es) de coordination}
\newabbreviation{ipa}{ipa}{infirmier(es) en pratique avancée}
\newabbreviation{ehpad}{ehpad}{établissement d'hébergement pour personnes âgées dépendantes}
\newabbreviation{adl}{adl}{Score ADL ( Activities of Daily Living) de \textsc{Katz}}
%
% Glossaire de statistiques
%
\newglossaryentry{alpha}{name={risque $\alpha$}, description={Probabilité de rejeter à tort l'hypothèse nulle càd conclure à une différence alors qu'il n'y en a pas.}}

\newglossaryentry{puissance}{name={puissance}, description={1-$\beta$, $\beta$ étant la probabilité de rejeter l'hypothèse nulle quand elle est fausse càd conclure à l'absence de différence alors qu'elle existe.}}

\newglossaryentry{likert}{name={Likert}, description={outil psychométrique permettant de mesurer une attitude chez des individus. Elle consiste en plusieurs items par lesquelles la personne interrogée exprime son degré d'accord ou de désaccord. Exemple : "tout à fait d'accord", "d'accord", "indifférent", "pas d'accord", "pas du tout d'accord".}}

\newglossaryentry{mediane}{name={médiane}, description={Valeur qui sépare la moitié inférieure et la moitié supérieure des termes d’une série statistique quantitative.}}

\newglossaryentry{quartiles}{name={quartiles}, description={Valeurs qui séparent une série statistique quantitative aux seuils de 25 \% et de 75 \% (la médiane sépare au seuil de 50 \%). Autrement dit la moitié des cas se situent entre les deux bornes des quartiles.}}

\newglossaryentry{nonparam}{name={non paramétrique}, description={Se dit d'une série numérique dont on ne peut affirmer qu'elle soit distribuée selon une loi normale (courbe en cloche de Gauss) ou des tests statistiques qu'on appliquera dans cette situation (test de Wilcoxon par ex.).}}

\newglossaryentry{discrete}{name={variables discrètes}, description={Variable non numérique consistant en un nombre réduit de classes. ex: Oui/Non, Pluie/Soleil/brouillard.}}

\newglossaryentry{bootstrap}{name={bootstrap}, description={Méthodes  basées sur la réplication multiple des données à partir du jeu de données étudié, selon les techniques de rééchantillonnage. Permet de simuler un gros échantillon à partir d'un petit. Utile pour estimer la taille nécessaire d'un échantillon.}}
